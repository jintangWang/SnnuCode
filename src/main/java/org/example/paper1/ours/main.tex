% This is samplepaper.tex, a sample chapter demonstrating the
% LLNCS macro package for Springer Computer Science proceedings;
% Version 2.21 of 2022/01/12
%
\documentclass[runningheads]{llncs}
%
\usepackage[T1]{fontenc}
% T1 fonts will be used to generate the final print and online PDFs,
% so please use T1 fonts in your manuscript whenever possible.
% Other font encondings may result in incorrect characters.
%
\usepackage{amsmath,amsfonts}
\usepackage{graphicx}
\usepackage{enumitem}
\setlength{\parindent}{1em}
% Used for displaying a sample figure. If possible, figure files should
% be included in EPS format.
%
% If you use the hyperref package, please uncomment the following two lines
% to display URLs in blue roman font according to Springer's eBook style:
%\usepackage{color}
%\renewcommand\UrlFont{\color{blue}\rmfamily}
%\urlstyle{rm}
%
\begin{document}
%
\title{Bilateral Access Control with Search for Pharmaceutical Data Sharing}
%
%\titlerunning{Abbreviated paper title}
% If the paper title is too long for the running head, you can set
% an abbreviated paper title here
%
\author{JinTang Wang\inst{1}, Tao Wang\inst{1}, ZhiChao Wang\inst{1}}

\authorrunning{J. Wang et al.}
% First names are abbreviated in the running head.
% If there are more than two authors, 'et al.' is used.
%
% TODO 修改
\institute{Shaanxi Normal University, Xi'an, Shaanxi 710062, P.R. China \and
Springer Heidelberg, Tiergartenstr. 17, 69121 Heidelberg, Germany
\email{lncs@springer.com}\\
\url{http://www.springer.com/gp/computer-science/lncs} \and
ABC Institute, Rupert-Karls-University Heidelberg, Heidelberg, Germany\\
\email{\{abc,lncs\}@uni-heidelberg.de}}
%
\maketitle              % typeset the header of the contribution
%
\section{Proposed Construction}
$\mathbf{Setup}(1^\lambda).$The algorithm generates a bilinear group $(\mathbb{G}, \mathbb{G}_T)$ of prime order $p$, with a generator $g \in \mathbb{G}$ and a bilinear map $e: \mathbb{G} \times \mathbb{G} \rightarrow \mathbb{G}_T$. It selects a random $\delta \in \mathbb{G}$ and computes $\delta' = \delta^z$, where $z \in \mathbb{Z}_p$ is chosen randomly. The algorithm chooses random values $\mu, \nu \in \mathbb{Z}_p$ and defines three hash functions:
$$
\mathcal{H}_1: \Omega_{\mathrm{snd}} \rightarrow \mathbb{G}, \quad \mathcal{H}_2: \Omega_{\mathrm{rcv}} \rightarrow \mathbb{G}, \quad \mathcal{H}_3: \{0,1\}^* \rightarrow \mathbb{G}
$$
where $\mathcal{H}_3$ is collision-resistant, and $\mathcal{H}_1, \mathcal{H}_2$ are modeled as random oracles. The algorithm outputs the master public key $mpk$ and the master secret key $msk$:
\begin{gather*}
mpk = (p, \mathbb{G}, \mathbb{G}_T, e, g, \delta, \delta', \mathcal{H}_1, \mathcal{H}_2, \mathcal{H}_3, e(g,g)^\mu, e(g,g)^\nu)\\
msk = (\mu, \nu, z)
\end{gather*}

$\mathbf{EKGen}(msk, \mathcal{S}).$Given the sender's attribute set $\mathcal{S} = (att_{\mathbf{snd},1}, \ldots, att_{\mathbf{snd},k})$, the encryption key generation algorithm randomly selects $\sigma \in \mathbb{Z}_p^*$ and computes:
$$ek_{1,i} = g^\mu \mathcal{H}_1(att_{\mathbf{snd},i})^\sigma, \quad ek_2 = g^\sigma \quad \text{for} \quad i \in [k]$$
The algorithm outputs the encryption key $ ek = (\mathcal{S}, \{ ek_{1,i} \}_{i \in [k]}, ek_2) $.

$\mathbf{DKGen}(msk, \mathbb{R}, bf)$
Given the access structure $ \mathbb{R} = (\mathbb{A}, \phi) $, where $ \mathbb{A} \in \mathbb{Z}_p^{l_\mathbb{A} \times n_\mathbb{A}} $ and the mapping $ \phi: [l_\mathbb{A}] \rightarrow \Omega_{\text{rcv}} $, the decryption key generation algorithm randomly selects $ \vec{v} = (\nu, v_2, \ldots, v_{n_\mathbb{A}})^\top $ and computes $ \vec{\omega} = \mathbb{A} \vec{v} $. For each $ i \in [l_\mathbb{A}] $, it randomly selects $ \tau_i \in \mathbb{Z}_p $ and computes:
$$
dk_{1,i} = g^{\omega_i} \mathcal{H}_2(\phi(i))^{\tau_i}, \quad dk_{2,i} = g^{\tau_i}
$$
Data user provides blinding factor $ \text{bf} = w \in \mathbb{Z}_p^* $, the algorithm computes the query key: $\text{QK} = \delta'^{\mu/w} \delta$. The algorithm outputs the decryption key: $dk = (\mathbb{R}, \{ dk_{1,i}, dk_{2,i} \}_{i \in [l_\mathbb{A}]}, \text{QK})$

$\mathbf{Encrypt}(ek, \mathcal{R}, \mathcal{S}', m, W).$ Given the receiver's attribute set $ \mathcal{R} = (att_{\mathbf{rcv},1}, \\ att_{\mathbf{rcv},2},\ldots, att_{\mathbf{rcv},l}) $ and the sender's attribute subset $\mathcal{S}'=(att_{\mathbf{snd},1}, att_{\mathbf{snd},2}, \\ \ldots, att_{\mathbf{snd},k^{\prime}})$, s.t. $ \mathcal{S}' \subseteq \mathcal{S} $, the encryption algorithm randomly selects $ s, \sigma', \tau \in \mathbb{Z}_p $ and computes:
\begin{gather*}
c_0 = m \cdot e(g, g)^{\nu s}, \quad c_1 = g^s, \quad c_{2,i} = \mathcal{H}_2(att_{\mathbf{rcv},i})^s,\\
c_3 = ek_2 \cdot g^{\sigma'} = g^{\sigma + \sigma'}, \quad c_4 = g^\tau
\end{gather*}
Let $ c_{1-4} = c_0 \| c_1 \| c_{2,1} \| \cdots \| c_{2,l} \| c_3 \| c_4 $. For each $ i' \in [k'] $, find $j$ such that $ att_{\mathbf{snd},i'} = att_{\mathbf{snd},j} $ and compute:
\begin{gather*}
ek_{1,i'} = ek_{1,j} \cdot \mathcal{H}_1(att_{\mathbf{snd},i'})^{\sigma'} = g^\mu \mathcal{H}_1(att_{\mathbf{snd},i'})^{\sigma + \sigma'},\\
c_{5,i'} = ek_{1,i'} \cdot \mathcal{H}_3(c_{1-4})^\tau = g^\mu \mathcal{H}_1(att_{\mathbf{snd},i'})^{\sigma + \sigma'} \mathcal{H}_3(c_{1-4})^\tau
\end{gather*}
For each keyword $ kw_i \in W $, compute the index: 
$$\kappa_i = e(g^\mu, \delta')^s \cdot e(g, \mathcal{H}_3(kw_i))^s$$
The algorithm outputs the keyword index $ \text{I}_{kw} = (\text{I}_1, \text{I}_2, \text{I}_3) $, where $ \text{I}_1 = c_1 = g^s $, $ \text{I}_2 = \delta^s $, and $ \text{I}_3 = \{ \mathcal{H}_3(\kappa_i) \}_{kw_i \in W} $. The ciphertext is:
$c = ((\mathcal{S}, \mathcal{R}), c_0, c_1, \{ c_{2,i} \}_{i \in [l]},$ $ c_3, c_4, \{ c_{5,i'} \}_{i' \in [k']}, \text{I}_{kw})$

$\mathbf{Verify}(\mathbb{S}, c).$ Given the sender's access structure $ \mathbb{S} = (\mathbb{M}, \rho) $, where $ \mathbb{M} \in \mathbb{Z}_p^{\ell_\mathbb{M} \times n_\mathbb{M}} $ and $ \rho: [\ell_\mathbb{M}] \rightarrow \Omega_{\text{snd}} $, let $ I = \{ i \in [\ell_\mathbb{M}] | \rho(i) \in \mathcal{S} \} $. Find the coefficients $ \{ \omega_i \}_{i \in I} $ such that $ \sum_{i \in I} \omega_i \mathbb{M}_i = (1, 0, \ldots, 0) $. Verify the equality:
$$
\prod_{i \in I} \left( \frac{e(c_{5,i}, g)}{e(\mathcal{H}_1(att_{\mathsf{snd},i}), c_3) \cdot e(\mathcal{H}_3(c_{1-4}), c_4)} \right)^{\kappa_i \omega_i} \overset{?}{=} e(g, g)^\mu
$$
If the equality holds, return 1; otherwise, return 0.

$\mathbf{Trapdoor}(mpk, \text{QK}, \text{bf}, kw).$ The trapdoor for keyword $kw$ is computed using the predefined blinding factor $ \text{bf} $ as follows:
$$T_1 = \mathcal{H}_3(kw) \cdot \text{QK}^{\text{bf}}, \quad T_2 = g^{\text{bf}}$$
The algorithm outputs the trapdoor $ T_{kw} = (T_1, T_2) $.

$\mathbf{Search}(\mathbb{S}, \text{I}_{kw}, T_{kw}, c)$. First, call $ \text{Verify}(\mathbb{S}, c) $. If the result is 0, output $ \perp $. Otherwise, compute:
$$
\kappa_{kw} = \frac{e(\text{I}_1, T_1)}{e(\text{I}_2, T_2)} = e(g^\mu, \delta')^s \cdot e(g, \mathcal{H}_3(kw))^s
$$
comparing $ \mathcal{H}_3(\kappa_{kw}) $ with the values in $ \text{I}_3 $, find the matching ciphertext. If no match is found, output $ \perp $; otherwise, return the matching $(c, \text{I}_{kw})$ to Receiver.

$\mathbf{TransKeyGen}(dk, mpk)$. The transform key generation algorithm first verifies that the input decryption key $dk$ satisfies the access structure $\mathbb{R}$ by checking if there exists a set of coefficients $\{\zeta_j\}_{j \in J}$ such that $\sum_{j \in J} \zeta_j \mathbb{A}_j = (1,0,\ldots,0)$ where $J = \{j \in [l_{\mathbb{A}}] | \phi(j) \in \mathcal{R}\}$. If verification fails, the algorithm outputs $\perp$. Otherwise, selects a unique secret value $\beta_u \in \mathbb{Z}_p^*$ for each user $u$ and for each component in $dk$, computes:
$$
tk_{1,i} = dk_{1,i}^{\beta_u},\quad tk_{2,i} = dk_{2,i}^{\beta_u} \quad \text{for} \quad i \in [l_\mathbb{A}]
$$
The algorithm outputs the transform key: $tk = (\mathbb{R}, \{tk_{1,i}, tk_{2,i}\}_{i \in [l_\mathbb{A}]})$

$\mathbf{Transform}(c, tk)$. The transform algorithm takes as input a ciphertext $c$ and transform key $tk$. Let $J = \{j \in [l_{\mathbb{A}}] | \phi(j) \in \mathcal{R}\}$. Find coefficients $\{\zeta_j\}_{j \in J}$ such that $\sum_{j \in J} \zeta_j \mathbb{A}_j = (1,0,\ldots,0)$. Compute partial decryption:
$$c'_0 = \prod_{j \in J} \left(\frac{e(tk_{2,j}, c_{2,j})}{e(tk_{1,j}, c_1)}\right)^{\zeta_j}$$
Output transformed ciphertext: $c' = (c_0, c'_0)$

$\mathbf{Decrypt}(c')$. The final decryption algorithm takes the transformed ciphertext $c'$ and performs:
$$m = c_0 \cdot {c'_0}^{1/\beta_u}$$

$\mathbf{Correctness}$.
If $ \mathcal{S} \models \mathbb{S} $, the ciphertext can pass the verification according to the verification algorithm:
$$
\begin{aligned}
&\prod_{i \in I} \left( \frac{e(c_{5,i}, g)}{e(\mathcal{H}_1(att_{\mathsf{snd},i}), c_3) \cdot e(\mathcal{H}_3(c_{1-4}), c_4)} \right)^{\kappa_i \omega_i} \\
&= \prod_{i \in I} \left( \frac{e(g^\mu \mathcal{H}_1(att_{\mathsf{snd},i})^{\sigma + \sigma'} \mathcal{H}_3(c_{1-4})^\tau, g)}{e(\mathcal{H}_1(att_{\mathsf{snd},i}), g^{\sigma + \sigma'}) \cdot e(\mathcal{H}_3(c_{1-4}), g^\tau)} \right)^{\kappa_i \omega_i} \\
&= \prod_{i \in I} e(g^\mu, g)^{\kappa_i \omega_i} = e(g, g)^{\mu \sum_{i \in I} \kappa_i \omega_i} = e(g, g)^\mu.
\end{aligned}
$$

If the keyword matches, the ciphertext can be retrieved through the Search algorithm by matching the keyword:
$$
\begin{aligned}
k_{kw} &= \frac{e(\text{I}_1, T_1)}{e(\text{I}_2, T_2)} = \frac{e(g^s, \mathcal{H}_3(kw) \text{QK}^w)}{e(\delta^s, g^w)} \\
&= \frac{e(g^s, \delta'^\mu \delta^w \mathcal{H}_3(kw))}{e(g, \delta)^{w s}} \\
&= \frac{e(g^s, \delta')^{\mu} e(g, \delta)^{w s} e(g, \mathcal{H}_3(kw))^s}{e(g, \delta)^{w s}} \\
&= e(g, \delta')^{\mu s} e(g, \mathcal{H}_3(kw))^s.
\end{aligned}
$$

Then, it checks if $H_{3}(k_{kw})$ appears in $\text{I}_3$ and outputs the matched ciphertext.

The correctness of the complete decryption process through transform can be verified as follows if $ \mathcal{R} \models \mathbb{R} $:
$$
\begin{aligned}
c_0^{\prime} & =\prod_{j\in J}\left(\frac{e(tk_{2,j}, c_{2,j})}{e(tk_{1,j}, c_1)}\right)^{\zeta_{j}} \\
& =\prod_{j\in J}\left(\frac{e(g^{\tau_{j}\beta_u},\mathcal{H}_2(att_{j})^{s})}{e(g^{\omega_{j}\beta_u}\mathcal{H}_2(\phi(j))^{\tau_{j}\beta_u},g^{s})}\right)^{\zeta_{j}}\\  
& =\prod_{j\in J}\left(\frac{{e(g,\mathcal{H}_2(att_{j}))^{s\tau_{j}\beta_u}}}{e(g,g)^{s\omega_{j}\beta_u}\cdot{e(\mathcal{H}_2(\phi(j)),g)^{s\tau_{j}\beta_u}}}\right)^{\zeta_{j}}\\  
& =\prod_{j\in J}e(g,g)^{-s\beta_u\omega_{j}\zeta_{j}}\\  
& =e(g,g)^{-s\beta_u\sum_{j\in J}\omega_{j}\zeta_{j}}\\  
& =e(g,g)^{-s\beta_u\nu}
\end{aligned}
$$

Finally, message $m$ can be recovered through: 
$$m = c_0 \cdot {c'_0}^{1/\beta_u} = \left( m \cdot e(g,g)^{\nu s} \right) \cdot e(g,g)^{-\nu s} = m$$

Therefore, our verification algorithm, search algorithm, and decryption algorithm are correct.

%
% ---- Bibliography ----
%
% BibTeX users should specify bibliography style 'splncs04'.
% References will then be sorted and formatted in the correct style.
%
\bibliographystyle{splncs04}
\bibliography{references}


\end{document}
