\documentclass[runningheads]{llncs}
\usepackage[T1]{fontenc}
\usepackage{amsmath,amsfonts}
\usepackage{graphicx}
\usepackage{enumitem}
\setlength{\parindent}{1em}
\usepackage{graphicx}
\begin{document}
\title{CFDS}

\section{Construction}
We propose an S-ABPRE-KU scheme in the key-policy attribute-based setting. Secret key is associated with access policy, and ciphertext is tagged with attribute set. As of, we use an LSSS access structure $(M,\rho)$ to represent a policy,$U$ to denote an attribute universe whereby $| U|$ is a polynomial in $1^\lambda .$ We use $KW$ to denote either a single keyword or a group of multiple keywords. Since $KW$ is the input of a hash function, if $KW$ represents a group of multiple keywords, it indicates that these keywords are with AND gates. We also note that $KW$ can be arbitrary $l$ength.

$1) \textit{ Setup}( 1^\lambda , U)$ .Run $(q,g,\hat{g},\mathbb{G}_1,\mathbb{G}_2,\mathbb{G}_T,e)\leftarrow BSetup(1^\lambda)$ .Choose $\alpha,\hat{\alpha},\delta,\epsilon_1,\epsilon_2,\epsilon_3,\beta_i,\chi\in_R\mathbb{Z}_q^*$,set
$h_i=g^{\beta_i},\hat{h}_i=\hat{g}^{\beta_i},t=g^\delta,\hat{t}=\hat{g}^\delta,z=g^\chi,\hat{z}=\hat{g}^\chi,f_1=g^{\epsilon_1},f_2=g^{\epsilon_2},f_3=g^{\epsilon_3},\hat{f}_1=\hat{g}^{\epsilon_1},\hat{f}_2=\hat{g}^{\epsilon_2}$,
$\hat{f}_3=\hat{g}^{\epsilon_3},i\in[1,|U|]$ .Choose Target Collision Resistant (TCR) hash functions [12]:
$H_1:\{0,1\}^\lambda\times\{0,1\}^\lambda\to\mathbb{Z}_q^*,H_2:\mathbb{G}_T\to\{0,1\}^{2\lambda},H_3:\{0,1\}^*\to\mathbb{Z}_q^*,H_4:\{0,1\}^*\to\mathbb{Z}_q^*$,
$H_5:\{0,1\}^\lambda\to\mathbb{Z}_q^*$ and $H_6:\{0,1\}^\lambda\to\{0,1\}^{poly(1^\lambda)}$ , a CCA-secure one-time symmetric key encryption $SY=(S.Enc,\bar{S.Dec}).$ The master secret key $msk=(\alpha,\hat{\alpha},\hat{z},\hat{t})$ ,the master public key $mpk=(g,\hat{g}$, $\{h_i,\hat{h}_i\}_{i\in[1,|U|]},f_1,f_2,f_3,\hat{f}_1,\hat{f}_2,\hat{f}_3,t,z,e(g,\hat{g})^\alpha,e(t,\hat{g})^{\hat{\alpha}},e(z,\hat{g})^{\hat{\alpha}},H_1,H_2,H_3,H_4,H_5,H_6,SY)$

$2) \textit{ KeyGen}( msk, ( M, \rho ) )$ .Let $M$ be an $l\times n$ matrix, and $\rho$ be the function that associates rows of $M$ to
attributes. Choose a random vector $(\vec{v})=(\alpha,y_2,\ldots,y_n)\in\mathbb{Z}_q^{*n}$ which to be used to share $\alpha$ .For $i=1$ to $l$ , compute $\phi_i=(\vec{v})\cdot M_i$ ,where $M_i$ is the vector related to $i$ -th row of $M$ .Choose $r_1,\ldots,r_l\in_R\mathbb{Z}_q^*$,and set $sk_{(M,\rho)}$ as

$$D_i=\hat{g}^{\phi_i}\cdot\hat{h}_{\rho(i)}^{r_i},R_i=\hat{g}^{r_i},\forall d\in\Gamma/\rho(i),Q_{i,d}=\hat{h}_d^{r_i},$$

where $i\in[1,l]$ , $\Gamma$ is the set of distinct attributes in $M\left(\mathrm{i.e.~}\Gamma=\{d:\exists i\in[1,l],\rho(i)=d\}\right.$ and $\Gamma/\rho(i)$
denotes the set $\Gamma$ with the element $\rho(i)$ removed if present. Note $sk_(M,\rho)$ implicitly includes $(M,\rho).$


$3) \textit{ Enc}( m, S, KW)$ .Set the original ciphertext $CT$ as

$$\begin{aligned}&A=(m||\sigma)\oplus H_2(e(g,\hat{g})^{\alpha s}),B=g^s,\{C_x=h_x^s\}_{x\in S},\\&D=e(t^{H_3(KW)}z,\hat{g})^{\hat{\alpha}s},E_1=f_1^s,\\&E_{2}=(f_2^{H_4(A,B,\{C_x\}_{x\in S},D,E_1)}f_3)^s,\end{aligned}$$

where $\sigma\in\{0,1\}^\lambda$ , $m\in\{0,1\}^\lambda$ and $s=H_1(m,\sigma)$ . $CT$ implicitly contains $S.$


$4) \textit{ Trapdoor}( msk, sk_{( M, \rho ) }, KW)$ .The trapdoor is generated by the collaboration between a secret key holder and

the fully trusted PKG.

· The PKG chooses a random vector $\vec{\mathbb{V}}=(\hat{\alpha},\hat{y}_2,\ldots,\hat{y}_n)\in\mathbb{Z}_q^{*n}$ which to be used to share $\hat{\alpha}$ .For $i=1$ to $l$ ,
it sets $\hat{\phi}_i=(\vec{\mathbb{V}})\cdot M_i$ .It further sends the following values to the user

$$\begin{aligned}\tau_{1,i}=&(\hat{t}^{H_3(KW)}\hat{z})^{\hat{\phi}_i}\cdot\hat{h}_{\rho(i)}^{\hat{r}_i},\tau_{2,i}=\hat{g}^{\hat{r}_i},\\\forall d\in&\Gamma/\rho(i),\tau_{3,i,d}=\hat{h}_d^{\hat{r}_i},\end{aligned}$$

where $\hat{r}_1,\ldots,\hat{r}_l\in_R\mathbb{Z}_q^*,i\in[1,l]$ , $\Gamma$ is the set of distinct attributes in $M.$
· The user re-randomizes the values, and sets the trapdoor token $\tau_KW$ as
$$\tau_{1,i}=\tau_{1,i}\cdot\hat{h}_{\rho(i)}^{\xi_i},\tau_{2,i}=\tau_{2,i}\cdot\hat{g}^{\xi_i},\\\forall d\in\Gamma/\rho(i),\tau_{3,i,d}=\tau_{3,i,d}\cdot\hat{h}_d^{\xi_i},$$
where $\xi_i\in_R\mathbb{Z}_q^*$ .Note $(M,\rho)$ implicitly includes in the token.

$_{5})Test(CT,\tau_{KW})$ .Parse $CT$ as $(S,A,B,\{C_x\}_{x\in S},D,E_1,E_2)$,and $\tau_KW$ as $(\tau_{1,i},\tau_{2,i},\tau_{3,i,d}).$ Suppose $S$
associated with $CT$ satisfies $(M,\rho)$ associated with $\tau_KW$ ,there exists a set of constants $\{w_i\}_{i\in I}\in\mathbb{Z}_q^*$ so that $\forall i\in I,\rho(i)\in S$ and$\sum_i\in Iw_iM_i=\left(1,0,\ldots,0\right)$.Given an original ciphertext $CT$ associated with a keyword set $KW^{\prime}$ and a token $\tau_KW$ ,one can verify that

$$\frac{e\left(B,\prod_{i\in I}\left(\tau_{1,i}\prod_{x\in\Delta/\rho(i)}\tau_{3,i,x}\right)^{w_i}\right)}{e\left(\prod_{x\in\Delta}C_x,\prod_{i\in I}\tau_{2,i}^{w_i}\right)}\overset{?}{\operatorname*{=}}D.$$

If the equation holds, i.e.

$$\begin{aligned}&\frac{e\left(B,\prod_{i\in I}\left(\tau_{1,i}\prod_{x\in\Delta/\rho(i)}\tau_{3,i,x}\right)^{w_i}\right)}{e\left(\prod_{x\in\Delta}C_x,\prod_{i\in I}\tau_{2,i}^{w_i}\right)}\\&=\frac{e\left(g^s,\prod_{i\in I}\left((t^{H_3(KW)}z)^{\hat{\phi}_i}\hat{h}_{\rho(i)}^{\hat{r}_i+\xi_i}\prod_{x\in\Delta/\rho(i)}\hat{h}_x^{\hat{r}_i+\xi_i}\right)^{w_i}\right)}{e\left(\prod_{x\in\Delta}h_x^s,\prod_{i\in I}(\hat{g}^{\hat{r}_i+\xi_i})^{w_i}\right)}\\&=e((t^{H_3(KW^{\prime})}z)^{\hat{\alpha}s},\hat{g}),\end{aligned}$$

it indicates that $KW=KW^{\prime}$ so that output 1, and output o otherwise. Similarly, if $CT$ is a re-encrypted
ciphertext, one verifies $\frac {e\left ( rk_6, \prod _{i\in I}\left ( \tau _{1, i}\prod _{x\in \Delta / \rho ( i) }\tau _{3, i, x}\right ) ^{w_i}\right ) }{e\left ( \prod _{x\in \Delta }rk_{7, x}, \prod _{i\in I}\tau _{2, i}^{w_i}\right ) }\overset {? }{\operatorname* { \operatorname* { = } } }rk_8$ .

6) $RKGen(sk_{(M,\rho)},S,KW)$ .Choose $\gamma\in_R\mathbb{Z}_q^*,\theta_1,\theta_2\in_R\{0,1\}^\lambda$ ,and set $rk$ as
$$rk_{1,i}=D_{i}^{{H_{5}(\theta_{1})}}\hat{f}_{1}^{\gamma},rk_{2}=\hat{g}^{\gamma},rk_{3,i}=R_{i}^{{H_{5}(\theta_{1})}},$$
$$rk_{4,i}=\forall d\in\Gamma/\rho(i)\mathrm{~}(Q_{i,d})^{H_5(\theta_1)},$$
$$rk_5=(\theta_1||\theta_2)\oplus H_2(e(g,\hat{g})^{\alpha\check{s}}),rk_6=g^{\check{s}},$$
$$rk_{7,x}=(h_x^{\breve{s}})_{x\in S},rk_8=e(t^{H_3(KW)}z,\hat{g})^{\hat{\alpha}\check{s}},$$
$$rk_{9}=(f_{2}^{{H_{4}(rk_{5},rk_{6},rk_{7,x},rk_{8})}}f_{3})^{{\check{s}}},$$


where $sk_{(M,\rho)}=\left(D_i,R_i,\forall d\in\Gamma/\rho(i)Q_{i,d}\right),i\in\left[1,l\right],\check{s}=H_1(\theta_1,\theta_2).$ Note $rk$ includes $(M,\rho)$ and $S.$

$7) ReEnc(CT,rk).$ Parse $CT$ as $(S, A, B, \{C_x\}_{x\in S}, D, E_1, E_2), rk$ as $((M,\rho), S', rk_{1,i}, rk_2, rk_{3,i}, rk_{4,i}, rk_5, rk_6, \{rk_{7,x}\}_{x\in S'}, rk_8, rk_9), i \in [1,l].$

i. Check the validity of $CT$ as

\begin{equation}
e(B,\hat{f}_1) \stackrel{?}{=} e(E_1,\hat{g})
\end{equation}
\\
\begin{equation}
e\left(\prod_{x\in S} C_x,\hat{g}\right) \stackrel{?}{=} e\left(B,\prod_{x\in S}\hat{h}_x\right)
\end{equation}
\\
\begin{equation}
e(B,\hat{f}_2^{H_4(A,B,\{C_x\}_{x\in S},D,E_1)}\hat{f}_3) \stackrel{?}{=} e(E_2,\hat{g})
\end{equation}

$\text{If one of the equations does not hold, output}\perp.\text{ Else, proceed.}$    

$\begin{aligned}\mathrm{ii.} & \mathrm{If~}S\text{ associated with }CT\mathrm{~satisfies~}(M,\rho)\text{ associated with }rk,\mathrm{~let~}I\subset\{1,2,\ldots,l\}\text{ be a set of indices and}\\  & \{w_{i}\}_{i\in I}\in\mathbb{Z}_{q}^{*}\text{ be a set of constants so that }\forall i\in I,\rho(i)\in S\mathrm{~and}\sum_{i\in I}w_{i}M_{i}=\left(1,0,\ldots,0\right),\mathrm{and}\\  & \operatorname{define}\Delta=\left\{x:\exists i\in I,\mathrm{~}\rho(i)=x\right\}.\mathrm{Compute}\\  & \Omega=\frac{e\left(B,\prod_{i\in I}\left(rk_{1,i}\prod_{x\in\Delta/\rho(i)}rk_{4,x}\right)^{w_i}\right)}{e\left(E_1,rk_2^{\sum_{i\in I}w_i}\right)e\left(\prod_{x\in\Delta}C_x,\prod_{i\in I}rk_{3,i}^{w_i}\right)}\\  & =\frac{e\left(g^s,\prod_{i\in I}\hat{g}^{\phi_iH_5(\theta_1)w_i}\hat{f}_1^{\gamma w_i}\prod_{x\in\Delta}\hat{h}_x^{w_iH_5(\theta_1)r_i}\right)}{e\left(f_1^s,\hat{g}^{\gamma\sum_{i\in I}w_i}\right)e\left(\prod_{x\in\Delta}h_x^s,\prod_{i\in I}\hat{g}^{r_iH_5(\theta_1)w_i}\right)}\\  & =e(g^{s},\hat{g}^{\alpha H_5(\theta_1)}).\end{aligned}$

$\begin{aligned}\mathrm{iii.}&\quad\text{Compute the re-encrypted ciphertext as}T_{1}=S.Enc(CT||\Omega,H_{6}(S.Key))\mathrm{~,}T_{2}=(rk_{5}\mathrm{~,}rk_{6}\mathrm{~,}\\&\quad\{rk_{7,x}\}_{x\in S^{\prime}},rk_{8}\mathrm{~,}rk_{9})\mathrm{~,}T_{3}=(T_{3,1}=(S.Key||\theta_{3})\oplus H_{2}(e(g,\hat{g})^{\alpha\tilde{s}})\mathrm{~,~}T_{3,2}=g^{\tilde{s}},T_{3,3,x}=(h_{x}^{\tilde{s}})_{x\in S^{\prime}},\\&\quad T_{3,4}=(f_{2}^{H_{4}(T_{3,1},T_{3,2},T_{3,3,x})}f_{3})^{\tilde{s}})\mathrm{~,~where~}S.Key,\theta_{3}\in_{R}\{0,1\}^{\lambda}\mathrm{~,~}\tilde{s}=H_{1}(S.Key,\theta_{3})\mathrm{~.}\end{aligned}$


$\begin{aligned}&\mathrm{8)}Dec(sk_{(M,\rho)},CT).\\&&\text{(1) If CT is the original ciphertext,}\\&&&\text{i. Verify Eq. (3). If Eq. (3) does not hold, output}\perp.\text{Otherwise, proceed}.\\&&&\text{ii. If }S\text{ associated with }CT\mathrm{~satisfies~}(M,\rho)\text{ associated with }sk\text{ , there exists a set of constants}\\&&&\{w_i\}_{i\in I}\in\mathbb{Z}_q^*\text{ so that }\forall i\in I,\rho(i)\in S\mathrm{~and}\sum_{i\in I}w_iM_i=(1,0,\ldots,0).\mathrm{~Compute}\\&&&e\left(B,\prod_{i\in I}\left(D_i\prod_{x\in\Delta/\rho(i)}Q_{i,x}\right)^{w_i}\right)/e\left(\prod_{x\in\Delta}C_x,\prod_{i\in I}R_i^{w_i}\right)\\&&&=\frac{e\left(g^s,\prod_{i\in I}\hat{g}^{\phi_iw_i}\left(\prod_{x\in\Delta}\hat{h}_x\right)^{r_iw_i}\right)}{e\left(\prod_{x\in\Delta}h_x^s,\prod_{i\in I}\hat{g}^{r_iw_i}\right)}\\&&&=e(g,\hat{g})^{\alpha s}.\end{aligned}$

$\begin{aligned}\text{iii. Output the message by computing }m||\sigma&=A\oplus H_{2}(e(g,\hat{g})^{\alpha s})\mathrm{~if~}B=g^{H_{1}(m,\sigma)},\\E_{2}&=(f_{2}^{H_{4}(A,B,\{C_{x}\}_{x\in S},D,E_{1})}f_{3})^{H_{1}(m,\sigma)}\mathrm{~and}\prod_{x\in S}C_{x}=\prod_{x\in S}h_{x}^{H_{1}(m,\sigma)}\text{ ; else, output}\perp.\end{aligned}$

(2) If $CT$ is the re-encrypted ciphertext,

i. Verify

\begin{equation}
e(T_{3,2}, \hat{f}_2^{H_A(T_{3,1},T_{3,2},T_{3,3},x)} \hat{f}_3) \stackrel{?}{=} e(T_{3,4},\hat{g})
\end{equation}
$\text{If the equation does not hold, output}\perp.\text{ Otherwise, proceed.}$
$\begin{aligned}&\text{ii. Compute}\\\\&&\frac{e\left(T_{3,2},\prod_{i\in I}\left(D_i\prod_{x\in\Delta/\rho(i)}Q_{i,x}\right)^{w_i}\right)}{e\left(\prod_{x\in\Delta}T_{3,3,x},\prod_{i\in I}R_i^{w_i}\right)}\\&&=e(g,\hat{g})^{\alpha\tilde{s}}.\end{aligned}$
and recover $S.Key$ by computing $S.Key||\theta_3=T_{3,1}\oplus H_2(e(g,\hat{g})^{\alpha\tilde{s}})$ .Proceed if
$T_{3,2}=g^{H_1(S.Key,\theta_3)},T_{3,4}=(f_2^{H_4(T_{3,1},T_{3,2},T_{3,3,x})}f_3)^{H_1(S.Key,\theta_3)}$ and $\prod_{x\in S}T_{3,3,x}=\prod_{x\in S}h_x^{H_1(S.Key,\theta_3)};$
otherwise, output $\perp.$
iii. Recover $\theta _1$ from $rk_5,rk_6$ ,$\{rk_7,x\}_{x\in S}$ ,$rk_8$ and $rk_9$ as above.
iv. Compute $CT||\Omega=S.Dec(T_1,H_6(S.Key))$ ,and further set $\Omega^{H_5(\theta_1)^{-1}}=e(g^s,\hat{g}^\alpha).$
v. If Eq.(3) does not hold, output $\bot$ . Otherwise, proceed. Compute $m||\sigma=A\oplus H_2(e(g,\hat{g})^{\alpha s})$ ,and then output the message $m$ if $B=g^H_1(m,\sigma)$ and $E_2=(f_2^{H_4(A,B,\{C_x\}_{x\in S},D,E_1)}f_3)^{H_1(m,\sigma)};$ otherwise, output$\perp.$

\end{document}
