\documentclass[runningheads]{llncs}
\usepackage[T1]{fontenc}
\usepackage{amsmath,amsfonts}
\usepackage{graphicx}
\usepackage{enumitem}
\setlength{\parindent}{1em}
\usepackage{graphicx}
\begin{document}
\title{CFDS}

\section{Construction}
1) $\textit{Setup}( \lambda )$ $\to$ $( msk, mpk, kpol) {: }$ Given a security parameter $\lambda$, the KGC generates a bilinear group $\Psi=$ $(p,\mathbb{G},\mathbb{G}_T,g,e).$ Specifically, $e$ denotes a bilinear mapping function $\mathbb{G}\times\mathbb{G}\to\mathbb{G}_T.$ Next, a hash function $H[\mathbb{G}_T]\to\mathcal{M}$ is generated by the KGC, where $\mathcal{M}$ represents the message space composed of $\{0,1\}^*.$ Let the size of user attribute universe $\mho=(attr_1,attr_2,\cdots,attr_n)$ equals to $n.$ The KGC picks $2n$ random values $(r_{p,1},\cdots,r_{p,n})\in\mathbb{Z}_p$ and $(r_{e,1},\cdots,r_{e,n})\in$
$$msk=(r_{p,1},\cdots,r_{p,n},r_{e,1},\cdots,r_{e,n}).$$
$\operatorname*{Next,\text{ the KGC randomly picks }}\alpha\in\mathbb{Z}_p$ and $\beta\in\mathbb{Z}_p$ at random
and generates the master preference key $k\textit{pol as follows. }$
$$kpol=(\alpha,\beta).$$
Finally, the KGC publishes the master public key mpk to all entities as follows.
$$mpk=(p,g,\mathbb{G},\mathbb{G}_T,e,g^\alpha,g^\beta,\{R_{e,i}=g^{r_{e,i}}\}_{i=1}^n,$$
$$\{R_{p,i}=g^{r_{p,i}}\}_{i=1}^n,H[\cdot]).$$

2) $EKGen( \{ msk, \sigma \} )$ $\to$ $ek_\sigma {: }$ The KGC verifies the sender's attribute set $\sigma=\{attr_i\}_{i=1}^\sigma$ and generates the sender's encryption keys $ek_\mathrm{\sigma}$ as follows.
$$ek_\sigma=\{g^{\frac\omega{r_{p,i}}}\}_{i=1}^\sigma,$$
where $\omega\in\mathbb{Z}_p$ is a unique user symbol, which can prevent user collusion and achieve efficient user revocation by removing the corresponding parameter $\omega.$ Next, the KGC returns the encryption key $ek_\mathrm{\sigma}$ to the sender and the user symbol $\omega$ to the cloud server.
Example $l:$ Assume the attribute universe $as$ $U= ( attr_1$, $attr_{2},attr_{3},attr_{4},attr_{5},attr_{6})and\sigma=(attr_{1},attr_{3},attr_{5})$, $and~(attr_{1}~=~Man,~attr_{2}~=~Woman,~attr_{3}~=~English$, $attr_{4}$ = $Chinese$, $attr_{4}$ = $Student$, $attr_{6}$ = $Teacher)$ $is$ $a$ case in point. Then, the KGC generates the encryption key $ek_{\sigma}$ as $\{g^{\frac{w}{r_{p,1}}},g^{\frac{w}{r_{p,3}}},g^{\frac{w}{r_{p,5}}}\}.$

3) $DkGen( \{ msk, \rho \} )$ $\to$ $dk_\rho {: }$ The KGC verifies the receiver's attribute set $\rho=\left\{attr_i\right\}_{i=1}^\rho$ and generates the decryption key $dk_\rho$ as follows.
$$dk_\rho=\{g^{\frac\gamma{r_{e,i}}}\}_{i=1}^\rho,$$
where $\gamma\in\mathbb{Z}_p$ is a unique user symbol. Similarly, the parameter $\gamma$ can be used to prevent user collusion and achieve efficient user revocation. Finally, the KGC returns the decryption key $dk_\rho$ to the receiver and sends $\gamma$ to the cloud server.
Example 2: Assume that $\rho = ( attr_{2}, attr_{4}, attr_{6})$ and the user symbol is $\gamma . \textit{Then, the KGC generates the decryption key}$ $dk_{\rho}$ as $\{g^\frac\gamma{r_{e,2}},g^{\frac\gamma{r_{e,4}}},g^{\frac\gamma{r_{e,6}}}\}.$

4) $PolGen( kpol, \mathbb{S} , \mathbb{R} ) \to ( ek_\mathbb{S} , dk_\mathbb{R} ) {: }$ The KGC combines

the secret key $msk=(\alpha,\beta)$ to generate the preference key.
$\textbf{Specifically, the KGC considers the following two cases. }$

$\textbf{ Sender.}$ If the user is a sender, the KGC combines the secret key $\alpha$ and the preference $\mathbb{S}=\left\{attr_i\right\}_{i=1}^{\mathbb{S}}$ to generate a $(|\mathbb{S}|-1)$-dimension random polynomial function $Q_1(x)$ as follows.
$$Q_1(x)=\alpha+a_1x+\cdots+a_{|\mathbb{S}|-1}x^{|\mathbb{S}|-1},$$
vhere $Q_1( 0)$ = $\alpha , | \mathbb{S} |$ represents the size of th oreference S, and $\{ a_i\} _{i= 1}^{| \mathbb{S} | - 1}$ represents polynomial coefff oreference S, and $\{ a_i\} _{i= 1}^{| \mathbb{S} | - 1}$ represents polynomial coefff oreference S, an$\{ a_i\} _{i= 1}^{| \mathbb{S} | - 1}$ repr cients. Subsequently, based on $Q_1(x)$, the KGC combines Newton's interpolation formula to compute the Newton parameters $\left\{n_{1,i}(0)\right\}_{i=1}^{|\mathbb{S}|}$ and $\left\{K_{1,i}\right\}_{i=1}^{|\mathbb{S}|}.$ Specifically, fon $i=1,2,\cdots,|\mathbb{S}|$, the KGC computes the Newton parameters as follows.
$$\begin{aligned}K_{1,i}&=\frac{f[\mathbb{S}_2,\mathbb{S}_3,\cdots,\mathbb{S}_i]-f[\mathbb{S}_1,\mathbb{S}_2,\cdots,\mathbb{S}_{i-1}]}{\mathbb{S}_i-\mathbb{S}_1},\\K_{1,1}&=f(\mathbb{S}_1),\\n_{1,i}(0)&=(-1)^{i-1}\Pi_{j=1}^{i-1}\mathbb{S}_j,\:n_{1,1}(0)=1.\end{aligned}$$
Next, the KGC converts the Newton parameters as $\left\{n_{1,i}(0)\right\}_{i=1}^{\mathbb{S}}$ and $\left\{K_{1,i}\right\}_{i=1}^{\mathbb{S}}$ by changing the indexes. Note that in this process, the values of Newton's parameters are not changed, and only the indexes are changed correspondingly. Finally, the KGC generates the sender's encryption key $ek_{\mathbb{S}}$ as follows and returns it to the sender.
$$ek_\mathbb{S}=\{g^{K_{1,i}\cdot n_{1,i}(0)\cdot r_{e,i}}\}_{i=1}^\mathbb{S}.$$
Example 3: Assume that $\mathbb{S}=(attr_{2},attr_{4},attr_{6}).Then$, the KGC generates the Newton coefficients as $\{K_{1,1},K_{1,2}$, $K_{1,3}\}$ and $\{n_{1,1}(0),n_{1,2}(0),n_{1,3}(0)\}.$ Next, the KGC changes the indexes of Newton's parameters as $\{ K_{1, 2}, K_{1, 4}, K_{1, 6}\} \textit{and}$ $\{ n_{1, 2}( 0) , n_{1, 4}( 0) , n_{1, 6}( 0) \} . \textit{ Finally, the encryption key eks is}$ generated as $\{g^K_{1,2}\cdot n_{1,2}(0)\cdot r_{e,2},g^{K_{1,4}\cdot n_{1,4}(0)\cdot r_{e,4}},g^{K_{1,6}\cdot n_{1,6}(0)\cdot r_{e,6}}\}.$

$\textbf{Receiver.}$ If the user is a receiver, the KGC combines the secret key $\beta$ and the preference $\mathbb{R}=\{attr_i\}_{i=1}^\mathbb{R}$to generate a $|\mathbb{R}|-1)$-dimension random polynomial function $Q_2(x)$ as follows.
$$Q_2(x)=\beta+b_1x+\cdots+b_{|\mathbb{R}|-1}x^{|\mathbb{R}|-1},$$
$\begin{array}{rcl}\mathrm{where~}Q_2(0)&=&\beta,|\mathbb{R}|&\text{represents the size of the}\\\mathrm{preference~}\mathbb{R},&\mathrm{and~}\{b_i\}_{i=1}^{|\mathbb{R}|-1}&\text{represents polynomial param}\end{array}$ eters. Subsequently, based on $Q_2(x)$, the KGC combines Newton's interpolation formula to compute Newtor arameters $\left\{n_{2,i}(0)\right\}_{i=1}^{|\mathbb{R}|}$ and $\left\{K_{2,i}\right\}_{i=1}^{|\mathbb{R}|}.$ Specifically, fon $i=1,2,\cdots,|\mathbb{R}|$, the KGC computes the Newton parameters $\left\{n_{2,i}(0)\right\}_{i=1}^{|\mathbb{R}|}$ and $\left\{K_{2,i}\right\}_{i=1}^{|\mathbb{R}|}$ as follows.
$$\begin{aligned}K_{2,i}&=\frac{f[\mathbb{R}_2,\mathbb{R}_3,\cdots,\mathbb{R}_i]-f[\mathbb{R}_1,\mathbb{R}_2,\cdots,\mathbb{R}_{i-1}]}{\mathbb{R}_i-\mathbb{R}_1},\\K_{2,1}&=f(\mathbb{R}_1),\\n_{2,i}(0)&=(-1)^{i-1}\Pi_{j=1}^{i-1}\mathbb{R}_j,\:n_{2,1}(0)=1.\end{aligned}$$
Similar to the encryption key generation, the KGC changes the indexes and converts the Newton parameters as $\{n_{2,i}(0)\}_{i=1}^\mathbb{R}$ and $\{K_{2,i}\}_{i=1}^\mathbb{R}.$ Next, the KGC generates the receiver's decryption key $ek_\mathbb{R}$ as follows and returns
it to the receiver.
$$ek_\mathbb{R}=\{g^{K_{2,i}\cdot n_{2,i}(0)\cdot r_{p,i}}\}_{i=1}^\mathbb{R}.$$
$Example4: Assume$ that $\mathbb{R}$ = $( attr_1$, $attr_3$, $attr_5) .$
Then, the KGC generates the Newton coeffcients as $\{K_{2,1}$,
$\begin{aligned}K_{2,3},\: K_{2,5}\}\: and\:\{n_{2,1}(0),\: n_{2,3}(0),\: n_{2,5}(0)\}\: and\: generates\end{aligned}$
the decryption key $dk_\mathbb{R}$ as $\{g^K_{2,1}\cdot n_{2,1}(0)\cdot r_{p,1},g^{K_{2,3}\cdot n_{2,3}(0)\cdot r_{p,3}}$,
$g^{K_{2,5}\cdot n_{2,5}(0)\cdot r_{p,5}}\}.$

5) $Enc( ek_\sigma , ek_\mathbb{S} , m) \to C{: \text{ The sender inputs the encryp- }}$ tion keys $(ek_\sigma,ek_\mathbb{S})$ and a message $m$ to generate the ciphertext $C$, where $m\in \mathcal{M} .$ Specifically, the sender performs the encryption phase as follows.
1. Pick random numbers $\{ r_i\} _{i= 1}^4.$ 2. Compute
$$\begin{aligned}&R_{i}=g^{r_i},(i=1,2,3,4),\\&c_{0}=m\oplus H[e(R_1,R_3)]\oplus H[e(R_2,R_4)],\end{aligned}$$
$$c_{1,i}=ek_{\sigma,i}^{r_1},(i\in\sigma),$$
$$c_{2,i}=ek_{\mathbb{S},i}^{r_2},(i\in\mathbb{S})$$,
$$c_3=e(R_1,R_3)\cdot e(g^\beta,R_1)$$,
$$c_4=e(R_2,R_4)\cdot e(g^\alpha,R_2),$$
$$c_5=\frac{(g^\alpha)^{r_2}}{(g^\beta)^{r_1}}=g^{\alpha\cdot r_2-\beta\cdot r_1}.$$
$\begin{array}{ll}3.&C=&(c_0,\{c_{1,i}\}_{i=1}^\sigma,\{c_{2,i}\}_{i=1}^\mathbb{S},c_3,c_4,c_5)\text{ is sent to the}\\&\text{cloud server.}\end{array}$

6) $TrGen( dk_\rho , dk_{\mathbb{S} }) \to T{: }$ The receiver takes the decryption keys $(dk_\rho,dk_\mathbb{S})$ as input, and generates the trapdoor $T=$ $\left\{\left\{T_{1,i}\right\}_{i=1}^{\mathbb{S}},\left\{T_{2,i}\right\}_{i=1}^{\rho},T_3\right\}.$ Specifıcally, a receiver generates the matching tags by the following operations. 1. Select a random number $t(t\in Z_p).$ 2. Generate the matching trapdoor as follows.
$$T_{1,i}=dk_{\mathbb{R},i}^t,(i\in\mathbb{R}),\:T_{2,i}=dk_{\rho,i}^t,\:(i\in\rho),\:T_3=g^t.$$
3. The matching trapdoor $T=\left\{\left\{T_{1,i}\right\}_{i=1}^{\mathbb{R}},\left\{T_{2,i}\right\}_{i=1}^{\rho},T_{3}\right\}$ is sent to the cloud server.

7) $Match( C, T)$ $\to$ $( b, \mathbb{S} _j, \mathbb{R} _j) ) {: }$ With the ciphertext $C$ and matching trapdoor $T$, the cloud server computes $\{c_{1,i}^*=$ $c_{1,i}^{\frac1\omega}\}_{i=1}^{\sigma}$ and $\{c_{2,i}^*=c_{2,i}^{\frac1\gamma}\}_{i=1}^{\mathbb{S}}.$ Next, the cloud server constructs a sequence $\mathbb{S}_j$ and $\mathbb{R}_j$ by selecting $|\mathbb{S}|$ and $|\mathbb{R}|$ elements from $\left\{T_{2,i}\right\}_{i=1}^\rho$ and $\left\{c_{1,i}\right\}_{i=1}^\sigma$, respectively. Subsequently, the cloud server performs the following computation.
$$\frac{\Pi_{i=1}^{\mathbb{S}_j}e(c_{2,i}^*,T_{2,i})}{\Pi_{i=1}^{\mathbb{R}_j}e(c_{1,i}^*,T_{1,i})}\overset{?}{\operatorname*{=}}e(T_3,c_5).$$
The above steps are iteratively executed by the cloud server until the equation holds or all element combinations in $\{c_{1,i}^*\}_{i=1}^\sigma$ and $\{T_{2,i}\}_{i=1}^\rho$ are tried. Specifically, when $(\mathbb{R}\subset\sigma)\cap$ $(\mathbb{S}\subset\rho)$ holds, the above equation holds, and the cloud server outputs $b=1.$ Finally, the cloud server returns the ciphertext $\begin{array}{l}C=(c_0,\{c_{1,i}^*\}_{i=1}^\sigma,\{c_{2,i}^*\}_{i=1}^\mathbb{R},c_3,c_4,c_5),\text{ the sequences }\mathbb{S}_j\\\mathbb{R}_j\:\text{to the receiver. Otherwise, }b=0\:\text{is outputted, and the}\end{array}$ cloud server utilizes other senders’ciphertexts to perform the $\textbf{Match}$ phase. Note that since a ciphertext can be matched by multiple trapdoors simultaneously, PriBAC supports concurrent requests from a batch of receivers, and even can speed up the matching process by introducing existing parallel processing techniques.

8) $Dec( dk_\rho , dk_\mathbb{S} , \mathbb{S} _j, \mathbb{R} _j, C)$ $\to m$ / $\perp :$ The receiver inputs the decryption keys $dk_\rho$ and $dk_\mathbb{S}$, the sequences $\mathbb{R}_j$ and $\mathbb{S}_j$, and the ciphertext $C.$ Subsequently, the receiver recovers the message $m.$ Specifically, when $(\mathbb{S}\subset\sigma)\cap(\mathbb{R}\subset\rho)$, the receiver can recover the message $m.$ Otherwise, it outputs an error symbol $\bot.$ Specifically, the receiver performs the decryption phase as follows.
1. Based on $\mathbb{S}_j$, the receiver computes
$$d_1=\frac{c_4}{\Pi_{i=1}^{\mathbb{S}_j}e(c_{2,i}^*,dk_\rho)}.$$
2. Based on $\mathbb{R}_j$, the receiver computes
$$d_2=\frac{c_3}{\Pi_i^{\mathbb{R}_j}e(c_{1,i}^*,dk_\mathbb{R})}.$$
3. Recover $m$ as $m=c_0\oplus H[d_1]\oplus H[d_2].$

\end{document}
